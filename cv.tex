\documentclass[11pt,a4paper]{article}

%========== Packages ==========
\usepackage[utf8]{inputenc}
\usepackage[T1]{fontenc}
\usepackage[margin=2.5cm]{geometry}
\usepackage{hyperref}
\usepackage{enumitem}
\usepackage{titlesec}
\usepackage{lmodern}
\usepackage{setspace}
\usepackage{parskip}
\usepackage{multicol}

%========== Style ==========
\setlength{\parindent}{0pt}
\setstretch{1.1}
\titleformat{\section}{\Large\bfseries\scshape}{}{0em}{}
\titleformat{\subsection}{\large\bfseries}{}{0em}{}
\renewcommand{\labelitemi}{--}
\setlist[itemize]{leftmargin=1.5em, itemsep=0.3em}
\setlist[enumerate]{leftmargin=1.5em, itemsep=0.6em}

%========== Document ==========
\begin{document}

%========== Header ==========
\begin{center}
    {\huge \textbf{Giacomo Fenzi}} \\[0.3em]
\end{center}

%========== Personal Information ==========
\section*{Personal Information}
\begin{tabular}{rl}
  \textbf{Email:} &\href{mailto:giacomo.fenzi@epfl.ch}{giacomo.fenzi@epfl.ch} \\
  \textbf{Website:} &\url{gfenzi.io} \\
  \textbf{Birth place \& date:} & Milan (Italy), 28 March 1998 \\
  \textbf{Nationality:} & Italian \\
  \textbf{Languages:} & English (fluent), Italian (native) \\
\end{tabular}

%========== Research Interests ==========
\section*{Research Interests}
Cryptography, Security, Complexity Theory

%========== Education ==========
\section*{Academic experience}
\begin{itemize}
  \item \textbf{EPFL, Ph.D.}, Computer Science, present (expected to graduate September 2026) \\
  Thesis: \emph{Query-Optimal Interactive Oracle Proofs from Code Switching} \\
  Advised by Alessandro Chiesa \\
  Attended the Simons Institute \emph{Cryptography 10 Years Later: Obfuscation, Proof Systems, and Secure Computation} program in May to August 2025

  \item \textbf{ETH Z\"urich, M.Sc.}, Cybersecurity, September 2022 \\
  Thesis: \emph{Klondike: Finding Gold in SIKE} \\
  Advised by Kenneth Paterson and Fernando Virdia

  \item \textbf{University of St.\ Andrews, B.Sc.}, Computer Science and Mathematics, July 2020 \\
  Thesis: \emph{Zero Knowledge Proofs Theory and Applications} \\
  Advised by Stephen Linton and Louis Theran
\end{itemize}

%========== Awards ==========
\section*{Awards and Grants}
\begin{itemize}
  \item Best Paper Award at CRYPTO 2024 (recognizes the paper ``STIR: Reed–Solomon Proximity Testing with Fewer Queries'' for a breakthrough result)
  \item Judge of ``The Proximity Prize'' (\$1M prize to forward research in long-standing coding theory conjectures.)
  \item Teaching Assistant Award 2024 at EPFL (awarded for exceptional contribution to teaching excellence)
  \item Ethereum Foundation Academic Grant (awarded \$80'000 for \emph{Post-quantum signatures for Ethereum from hash-based succinct arguments}).
  \item Academic Research Program Grant (awarded \$25'000 for \emph{Small and efficient multilinear polynomial commitments in the pure ROM}).
\end{itemize}

%========== Publications ==========
\section*{Publications}

\subsection*{2025}
\begin{enumerate}
  \item \textit{Linear Time Accumulation Schemes}\\
  Benedikt Bünz, Alessandro Chiesa, Giacomo Fenzi, William Wang\\
  \textbf{TCC 2025} (23rd Theory of Cryptography Conference)
  
  \item \textit{Time-Space Trade-offs for the Sumcheck Protocol}\\
  Anubhav Baweja, Alessandro Chiesa, Elisabetta Fedele, Giacomo Fenzi, Pratyush Mishra, Tushar Mopuri, Andrew Zitek-Estrada\\
  \textbf{TCC 2025} (23rd Theory of Cryptography Conference)  

  \item \textit{TensorSwitch: Nearly Optimal Polynomial Commitments from
Tensor Codes}\\
  Benedikt Bünz, Giacomo Fenzi, Ron Rothblum, William Wang\\
  \textbf{In submission}

  \item \textit{Zero-Knowledge IOPPs for Constrained Interleaved Codes} \\
  Alessandro Chiesa, Giacomo Fenzi, Guy Weissenberg\\
  \textbf{In submission}

  \item \textit{Small-field hash-based SNARGs are less sound than conjectured}\\
  Giacomo Fenzi, Antonio Sanso\\
  \textbf{In submission}

  \item \textit{Interactive Proofs for Batch Polynomial Evaluation}\\
  Gal Arnon, Alessandro Chiesa, Giacomo Fenzi, Eylon Yogev\\
  \textbf{In submission}
  
  \item \textit{zip: Reducing Proof Sizes for Hash-Based SNARGs}\\
  Giacomo Fenzi, Yuwen Zhang\\
  \textbf{In submission}
\end{enumerate}

\subsection*{2024}
\begin{enumerate}
  \item \textit{WHIR: Reed–Solomon Proximity Testing with Super-Fast Verification}\\
  Gal Arnon, Alessandro Chiesa, Giacomo Fenzi, Eylon Yogev\\
  \textbf{EUROCRYPT 2025} (44th Conference on the Theory and Applications of Cryptographic Techniques)
  
  \item \textit{zkSNARKs in the ROM with Unconditional UC-Security}\\
  Alessandro Chiesa, Giacomo Fenzi\\
  \textbf{TCC 2024} (22nd Theory of Cryptography Conference)
  
  \item \textit{STIR: Reed–Solomon Proximity Testing with Fewer Queries}\\
  Gal Arnon, Alessandro Chiesa, Giacomo Fenzi, Eylon Yogev\\
  \textbf{CRYPTO 2024} (44th International Cryptology Conference) \\
  \textbf{Best Paper Award} (awarded to two papers in this conference)
  
  \item \textit{Lova: Lattice-Based Folding Scheme from Unstructured Lattices}\\
  Giacomo Fenzi, Christian Knabenhans, Khanh Ngoc Nguyen, Duc Tu Pham\\
  \textbf{ASIACRYPT 2024} (30th International Conference on the Theory and Application of Cryptology and Information Security)
    
  \item \textit{Finding Bugs and Features Using Cryptographically-Informed Functional Testing}\\
  Giacomo Fenzi, Jan Gilcher, Fernando Virdia\\
   \textbf{RWC 2026} (Real World Crypto Symposium) \& \textbf{TCHES 2026} (Transactions on Crypthographic Hardware and Embedded Systems)
  
  \item \textit{Post-Quantum Access Control with Application to Secure Data Retrieval}\\
  Behzad Abdolmaleki, Hannes Blümel, Tianxiang Dai, Giacomo Fenzi, Homa Khajeh, Stefan Köpsell, Maryam Zarezadeh\\
  \textbf{Communications in Cryptology 2025}
\end{enumerate}

\subsection*{2023}
\begin{enumerate}
  \item \textit{SLAP: Succinct Lattice-Based Polynomial Commitments from Standard Assumptions}\\
  Martin R Albrecht, Giacomo Fenzi, Oleksandra Lapiha, Ngoc Khanh Nguyen\\
  \textbf{EUROCRYPT 2024} (44th Conference on the Theory and Applications of Cryptographic Techniques)
  
  \item \textit{Lattice-Based Polynomial Commitments: Towards Concrete and Asymptotical Efficiency}\\
  Giacomo Fenzi, Hossein Moghaddas, Ngoc Khanh Nguyen\\
  \textbf{Journal of Cryptology 2024}
\end{enumerate}

\subsection*{Books}
\begin{enumerate}
  \item \textit{Latin Diachronic Frequency Dictionary Vol. 2.}\\
  Giacomo Fenzi.\\
  \textbf{Propylaeum: Digital Classics Books, Universitätsbibliothek Heidelberg, 2022.}
  
  \item \textit{Latin Diachronic Frequency Dictionary Vol. 4.}\\
  Giacomo Fenzi, James Leslie, William Short, Thomas Spinelli.\\
  \textbf{Propylaeum: Digital Classics Books, Universitätsbibliothek Heidelberg, 2022.}
\end{enumerate}

%========== Work Experience ==========
\section*{Work Experience}

\begin{itemize}[leftmargin=1.5em]

  \item \textbf{Succinct Labs}, San Francisco, United States\\
  \textit{Research Intern}, \hfill May 2025 -- September 2025

  \item \textbf{ETH Z\"urich}, Z\"urich, Switzerland\\
  \textit{Hilfsassistent}, \hfill September 2021 -- August 2022

  \item \textbf{Twitter}, London, United Kingdom\\
  \textit{Intern, Make Everything Searchable}, \hfill June 2021 -- September 2021

  \item \textbf{University of St Andrews}, St Andrews, United Kingdom\\
  \textit{Research Assistant}, \hfill June 2020 -- August 2020

  \item \textbf{Goldman Sachs}, London, United Kingdom\\
  \textit{Intern, Merchant Banking Division}, \hfill June 2019 -- August 2019

  \item \textbf{Deloitte}, Milan, Italy\\
  \textit{Intern, Advanced Analytics}, \hfill June 2018 -- July 2018

  \item \textbf{TecGlass Digital}, Lalín, Spain\\
  \textit{Intern}, \hfill June 2017
\end{itemize}

\section*{Talks and Invited Lectures}

\emph{Hash-based succinct arguments}
\begin{itemize}[leftmargin=2em]
  \item 2025--10--04, Ethereum Foundation Summit on PQ Interop (Cambridge, UK)
  \item 2025--08--13, Google ZK \& AI Summit (Sunnyvale, CA, USA)
  \item 2025--06--20, Cryptography: 10 Years Later (Simons Institute, Berkeley, CA, USA)
  \item 2024--10--30, BSA EPFL (Lausanne, Switzerland)
\end{itemize}

\emph{Linear time accumulation schemes}
\begin{itemize}[leftmargin=2em]
  \item 2025--06--02, Stanford University (Stanford, USA)
  \item 2025--05--04, CAW (Madrid, Spain)
\end{itemize}

\emph{WHIR: Reed–Solomon Proximity Testing with Super--Fast Verification}
\begin{itemize}[leftmargin=2em]
  \item 2025--10--04, Ethereum Foundation Summit on PQ Interop (Cambridge, UK)
  \item 2025--05--04, EUROCRYPT25 (Madrid, Spain)
  \item 2025--03--31, Security and Privacy Lab (University of Pennsylvania, Online)
  \item 2025--03--23, ZKProof7 (Sofia, Bulgaria)
  \item 2025--02--18, Real World Crypto (Paris, France)
  \item 2024--11--28, IBM Research (Zurich, Switzerland)
  \item 2024--10--16, Nexus Speaker Series and Sumcheck Builder Group (Online)
\end{itemize}

\emph{STIR: Reed–Solomon Proximity Testing with Fewer Queries}
\begin{itemize}[leftmargin=2em]
  \item 2024--10--8, Nexus Speaker Series and Sumcheck Builder Group (Online)
  \item 2024--09--12, TUM Blockchain Conference (Munich, Germany)
  \item 2024--09--02, Swiss Crypto Day (St Gallen, Switzerland)
  \item 2024--07--03, Starkitecture (Online)
  \item 2024--05--22, ZKProof6 (Berlin, Germany)
  \item 2024--04--08, zkSummit11 (Athens, Greece)
\end{itemize}

\emph{UC--secure zkSNARKs}
\begin{itemize}[leftmargin=2em]
  \item 2024--12--02, TCC24 (Milan, Italy)
  \item 2024--05--22, ZKProof6 (Berlin, Germany)
  \item 2024--03--11, NYU (New York, NY, USA)
  \item 2024--03--08, University of Pennsylvania (Philadelphia, PA, USA)
  \item 2024--03--04, BUSec (Boston, MA, USA)
\end{itemize}

\emph{Lova: Lattice--Based Folding Scheme from Unstructured Lattices}
\begin{itemize}[leftmargin=2em]
  \item 2025--03--17, ENSL/CWI/KCL/IRISA Joint Cryptography Seminars (Online)
\end{itemize}

\emph{SLAP: Succinct Lattice--Based Polynomial Commitments from Standard Assumptions}
\begin{itemize}[leftmargin=2em]
  \item 2024--05--26, EUROCRYPT24 (Zurich, Switzerland)
  \item 2024--05--22, ZKProof6 (Berlin, Germany)
  \item 2023--12--15, CWI (Amsterdam, Netherlands)
\end{itemize}

\emph{Lattice--Based Polynomial Commitments: Towards Concrete and Asymptotical Efficiency}
\begin{itemize}[leftmargin=2em]
  \item 2025--07--06, ArcticCrypt25 (Longyearbyen, Svalbard)
\end{itemize}

\end{document}
