\documentclass[a4paper]{article}
    \usepackage{fullpage}
    \usepackage{amsmath}
    \usepackage{amssymb}
    \usepackage{textcomp}
    \usepackage{hyperref}
    \usepackage[utf8]{inputenc}
    \usepackage[T1]{fontenc}
    \usepackage[margin=1in]{geometry}
    \textheight=10in
    \pagestyle{empty}
    \raggedright

    %\renewcommand{\encodingdefault}{cg}
%\renewcommand{\rmdefault}{lgrcmr}

\def\bull{\vrule height 0.8ex width .7ex depth -.1ex }

% DEFINITIONS FOR RESUME %%%%%%%%%%%%%%%%%%%%%%%

\newcommand{\area} [2] {
    \vspace*{-9pt}
    \begin{verse}
        \textbf{#1}   #2
    \end{verse}
}

\newcommand{\lineunder} {
    \vspace*{-8pt} \\
    \hspace*{-18pt} \hrulefill \\
}

\newcommand{\header} [1] {
    {\hspace*{-18pt}\vspace*{6pt} \textsc{#1}}
    \vspace*{-6pt} \lineunder
}

\newcommand{\employer} [3] {
    { \textbf{#1} (#2)\\ \underline{\textbf{\emph{#3}}}\\  }
}

\newcommand{\contact} [3] {
    \vspace*{-10pt}
    \begin{center}
        {\Huge \scshape {#1}}\\
        #2 \\ #3
    \end{center}
    \vspace*{-8pt}
}

\newenvironment{achievements}{
    \begin{list}
        {$\bullet$}{\topsep 0pt \itemsep -2pt}}{\vspace*{4pt}
    \end{list}
}

\newcommand{\schoolwithcourses} [4] {
    \textbf{#1} #2 $\bullet$ #3\\
    #4 \\
    \vspace*{5pt}
}

\newcommand{\school} [4] {
    \textbf{#1} #2 $\bullet$ #3\\
    #4 \\
}
% END RESUME DEFINITIONS %%%%%%%%%%%%%%%%%%%%%%%

    \begin{document}
\vspace*{-40pt}

    

%==== Profile ====%
\vspace*{-10pt}
\begin{center}
	{\Huge \scshape {Giacomo Fenzi}}\\
	Milan, Italy $\cdot$ giacomofenzi@outlook $\cdot$ +44 (0) 7778055234\\
\end{center}

%==== Education ====%
\header{Education}
\textbf{St. Andrews University}\hfill St. Andrews, United Kingdom\\
BSc Mathematics \& Computer Science \hfill Sep 2016 - May 2020\\
\textit{Expect to graduate with a} First. \textbf{GPA}: 18.1/20 \\
\vspace{2mm}

\textbf{Worth School}\hfill Turners Hills, United Kingdom\\
IB Diploma (39/45) \hfill Sep 2014 - May 2016 \\

\vspace{2mm}

%==== Experience ====%
\header{Experience}
\vspace{1mm}

\textbf{Goldman Sachs} \hfill London, United Kingdom \\
\textit{Summer Intern, MBD Strat} \hfill{June 2019 | August 2019}
\vspace{-1mm}
\begin{itemize} \itemsep 1pt
    \item Developed mathematical models to solve linear constraint optimization problems in order to maximize leverage on investments.
    \item Designed a general indexing service to allow for configurable and efficient querying of a variety of company-owned data sources. 
    \item Delivered a full stack solution for automating collection of regulatory documents, and integrating it with the company's workflow solution and existing data solutions.  
\end{itemize}

\textbf{Deloitte} \hfill Milan, Italy\\
\textit{Advanced Analytic Consulting Intern} \hfill June 2018 | July 2018\\
\vspace{-1mm}
\begin{itemize} \itemsep 1pt
	\item Developed tools for Statistical Analysis and Forecasting clients' employee flow, using a combination of statistical methods such as ARIMA and machine learning frameworks such as XgBoost and TensorFlow. 
	\item Delivered a portfolio website to showcase projects of the Advanced Analytic Team, using a combination of D3.js and Bootstrap.
	\item Engineered a Telegram Bot to allow user to query Qlik dashboards and gather quantitative information on the fly. 
\end{itemize}

\textbf{Goldman Sachs} \hfill London, United Kingdom\\
\textit{Spring Intern, Engineering} \hfill April 2018\\
\begin{itemize} \itemsep 1pt
	\item Designed a system leveraging preexisting data sources in order to expose a API for companies fundamentals data, to be used within the Investement Management Division.  
\end{itemize}

\textbf{TecGlass Digital} \hfill Lalín, Spain\\
\textit{Summer Intern} \hfill June 2017\\
\begin{itemize} \itemsep 1pt
	\item Built from the ground up WPF applications to be used routinely by both the marketing team and the R\&D department, using .NET and C\#.
\end{itemize}


\header{Skills}
\begin{tabular}{ l l }
	Programming Languages: & Rust, Go, C++, C, Java, Python, JavaScript, Haskell \\
	Technologies:          & Docker, Tensorflow, XgBoost, RabbitMQ, Angular \\
	Research Interests: & Quantum Computing, Cryptography, Programming Language Design \\
	Languages:             & Italian, English  \textit{(Advanced)} | French, Spanish \textit{(Basic)} \\
\end{tabular}
\vspace{2mm}



\header{Awards}
\textbf{Dean\textquotesingle{}s List} \hfill St. Andrews University\\
Annual award for academic excellence, awarded three years in a row \hfill 2017-2018-2019 \\
\textbf{Headmaster\textquotesingle{}s Outstanding Achievement for Mathematics} \hfill Worth School\\
Highest possible honour awarded at the school, gained for Mathematical prowess \hfill 2016 \\
\vspace*{2mm}

\vbox{
\header{Publications}

{\textit{The Latin Diachronic Frequency Dictionary}}  \\
\text{ Spinelli, T., Short, W., Fenzi, G., Leslie, J. } \hfill  BRILL (forthcoming) \\
(The book has undergone peer-review and has been accepted for publication pending minor revision. Draft and documentation available). \\
This book, which is based on the Latin Diachronic Database project (Spinelli/Fenzi 2019; \href{https://risweb.st-andrews.ac.uk/portal/en/datasets/the-latin-diachronic-database-project(05bf041f-9654-4173-8c8a-b93a2efa0926).html}{
https://risweb.st-andrews.ac.uk/portal/en/datasets/the-latin-diachronic-database-project(05bf041f-9654-4173-8c8a-b93a2efa0926).html}, provides the first chronological and statistical analysis of the frequency of Latin lemmas as attested in a corpus of 9.5 million words and 307 Latin authors (4th BCE-6th CE).
}
\vspace*{2mm}

\vbox{
{\textbf{The Latin Diachronic Database Project}}  \\
\text{ Spinelli, T., Fenzi, G.  } \hfill  \href{https://zenodo.org/record/2562829}{doi:10.5281/zenodo.2562829} \\
This project aims to create an innovative toolkit for the quantitative computational analysis of the Latin language as well as to support and further enhance the digital study of ancient intertextuality. \\
\vspace*{2mm}
}

\vbox{
{\textbf{The First Online Dictionary of Latin Near-Synonyms}} \\
Spinelli, T., Fenzi, G. \hfill \href{https://risweb.st-andrews.ac.uk/portal/en/datasets/the-first-online-dictionary-of-latin-nearsynonyms(3cf644e6-86b8-44d0-a50a-b33c7ca86072).html}{doi:10.17630/3cf644e6-86b8-44d0-a50a-b33c7ca86072}\\
The First Online Dictionary of Latin Near-Synonyms is a digital humanities project aimed at providing students and researchers with the first modern monolingual dictionary of Latin near-synonyms. \\
}

\header{Presentations}

\textbf{Quantum Computing} \hfill University of St. Andrews \\
\textit{Teach Me X} \hfill{27 February 2020}
\vspace{-1mm}
\begin{itemize} \itemsep 1pt
    \item Introduction to Quantum Computing, starting from the basic mathematics, developing understanding of the quantum physical framework
    and culminating with an overview of the Quantum Fourier Transform
    \item Slides available \href{https://github.com/WizardOfMenlo/QuantumPresentationTMX}{here (https://linktr.ee/giacomo.fenzi)} 
\end{itemize}

\textbf{Rust and Safe Systems Programming} \hfill University of St. Andrews \\
\textit{Teach Me X} \hfill{13 February 2019}
\vspace{-1mm}
\begin{itemize} \itemsep 1pt
    \item Presentation about Rust, systems programming in general, and in particular new techniques to mitigate memory unsafety
    \item Slides available \href{https://docs.google.com/presentation/d/1ui4ByY8qFhqAsdYoyPBPzhyg8AbV5YJKAN9sNwv5QPw/edit?usp=sharing}{here (https://linktr.ee/giacomo.fenzi)} 
\end{itemize}

\end{document}